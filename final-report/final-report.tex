\documentclass[12pt]{article}
\usepackage{titlesec}
\usepackage{fullpage}
\usepackage{graphicx}
\graphicspath{{./}}
\usepackage{hyperref}
\hypersetup{
    colorlinks=true,
    linkcolor=blue,
    urlcolor=blue
}

\renewcommand{\maketitle}{
    \begin{center}
        {\Huge\bfseries Compiler Design for Lua\\[1em] Final Report}\\[12em]
    {\LARGE Submitted by}\\[5em]
    \begin{itemize}
        \setlength{\itemsep}{1.2\baselineskip}
        \raggedright
        \item{\Large Aritra Basu - 160905126 - Roll no. 17}\\
        \item{\Large Pavan Kalyan - 160905138 - Roll no. 22}\\
        \item{\Large Sai Vignesh - 160905142 - Roll no. 23}\\
    \end{itemize}
    \end{center}
}
\titleformat{\section}
{\LARGE\bfseries\raggedright}
{}
{0em}
{}

\titleformat{\subsection}
{\large\bfseries}
{}
{0em}
{}

\titleformat{\subsubsection}
{\normalsize\bfseries}
{}
{0em}
{}

\begin{document}
\maketitle
\begin{figure}[b]
    \centering
\includegraphics[scale=0.2]{mulogo}
\end{figure}
\newpage
\section{Introduction}
Lua is an extension programming language designed to support general procedural programming with data description facilities.\\ It also offers good support for object-oriented programming, functional programming, and data-driven programming.\\ Lua is intended to be used as a powerful, light-weight scripting language for any program that needs one. Lua is implemented as a library, written in clean C (that is, in the common subset of ANSI C and C++).
\section{Objective}
To create a Bottom up parser for Lua using Flex and Bison.\\
We use Flex to create a powerful lexical analyzer and Bison to create a Bottom-up Parser with proper and useful error messages. 
\section{BNF Grammar for Lua}
\begin{verbatim}
chunk ::= {stat [";"]} [laststat [";"]]

	block ::= chunk

	stat ::=  varlist "=" explist | 
		 functioncall | 
		 do block end | 
		 while exp do block end | 
		 repeat block until exp | 
		 if exp then block {elseif exp then block} [else block] end | 
		 for Name "=" exp "," exp ["," exp] do block end | 
		 for namelist in explist do block end | 
		 function funcname funcbody | 
		 local function Name funcbody | 
		 local namelist ["=" explist] 

	laststat ::= return [explist] | break

	funcname ::= Name {"." Name} [":" Name]

	varlist ::= var {"," var}

	var ::=  Name | prefixexp "[" exp "]" | prefixexp "." Name 

	namelist ::= Name {"," Name}

	explist ::= {exp ","} exp

	exp ::=  nil | false | true | Number | String | "..." | function | 
		 prefixexp | tableconstructor | exp binop exp | unop exp 

	prefixexp ::= var | functioncall | "(" exp ")"

	functioncall ::=  prefixexp args | prefixexp ":" Name args 

	args ::=  "(" [explist] ")" | tableconstructor | String 

	function ::= function funcbody

	funcbody ::= "(" [parlist] ")" block end

	parlist ::= namelist ["," "..."] | "..."

	tableconstructor ::= "{" [fieldlist] "}"

	fieldlist ::= field {fieldsep field} [fieldsep]

	field ::= "[" exp "]" "=" exp | Name "=" exp | exp

	fieldsep ::= "," | ";"

	binop ::= "+" | "-" | "*" | "/" | "^" | "%" | ".." | 
		 "<" | "<=" | ">" | ">=" | "==" | "~=" | 
		 and | or

	unop ::= "-" | not | "#"

\end{verbatim}

\section{Language used for implementation}
Flex was used for the Lexical analyzer.\\
Bison was used for the Bottom up parser.\\
C was used for the symbol table.\\
\section{Methodology}


\section{Documentation}


\section{Code}


\section{Sample Input and Output}


\end{document}
